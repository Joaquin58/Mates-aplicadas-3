\chapter*{ANTON-BIVENS-DAVIS 4 (Review Exercises) EJERCICIO 72}
\textbf{72)} De acuerdo con la Ley de Kepler, los planetas en nuestro Sistema Solar  se mueven en orbitas elípticas alrededor del Sol.\\
Si el acercamiento más cercano al Sol de un planeta ocurre en el tiempo $t=0$, entonces la distancia $r$ del centro del planeta al centro del Sol en algún tiempo posterior $t$ puede ser determinado por la ecuación:
\[
r = a(1-ecos \phi)
\]
donde  $a$ es la distancia promedio entre centros, $e$ es una constante postiva que mide la planitud de la orbita elíptica y $\phi$ es la solución a la ecuación de Kepler:
\[
\frac{2\pi t}{T} = \phi - e sen\phi
\]
en donde $T$ es el tiempo que tarda una orbita completa de el planeta. Calcula la distancia de la Tierra al Sol cuado $T=90$ días (Primero encuetra $\phi$ de la Ecuación de Kepler, y luego usa este valor de $\phi$ para encontrar la distancia. Usa $a = 150 \times 10^{6}$ km) \\

Ya que queremos encontrar  $r$ cuando el tiempo es de 90 días, para esto usaremos la fórmula $r = a(1-ecos \phi)$, nos hace falta el valor de $\phi$, para esto utilizaremos la fórmula $\frac{2\pi t}{T} = \phi - e sen\phi$. Aquí nos vemos en una complicación para hallar el valor de $\phi$, debemos ocupar el método de Newton.
\begin{align*}
    \frac{2\pi t}{T} = \phi - e sen\phi \\
    e sen \phi + \frac{2\pi t}{T} = \phi \\
     e sen \phi - \phi + \frac{2\pi t}{T} = 0 \\
\end{align*}
Sabemos que el Valor de $\frac{2\pi t}{T}$ es una constante en esta ecuación, esta vale aproximadamente $1.54927$. \\
\newline
Debemos recordar el Método de Newton:
\[
x_{n+1} = x_{n} - \frac{f(x_n)}{f'(x_n)}
\]
Ahora derivaremos $e sen \phi - \phi + \frac{2\pi t}{T} = 0$ para poder usar este método, derivaremos respecto a $\phi$.
\begin{align*}
    f(\phi) &= e sen \phi - \phi + \frac{2\pi t}{T} \\
    \frac{d}{d\phi} f(\phi) &= \frac{d}{d\phi} e sen \phi - \phi + \frac{2\pi t}{T} \\
    f'(\phi) &= e\frac{d}{d\phi} sen \phi - \frac{d}{d\phi} \phi + \frac{d}{d\phi} \frac{2\pi t}{T} \\ 
    \text{e y $\frac{2\pi t}{T}$ son contantes} \\
    f'(\phi) &= e cos \phi - 1 \\ 
\end{align*}
Ahora tomando como nuestra  primer aproximación de x a $\phi_1 = 1$.\\
Aplicaremos ahora el Método de Newton para tener una mejor aproximación.\\
\newline
Tenemos a $e=0.0167$ y a $\frac{2\pi t}{T}=1.54927$, así:\\
\begin{align*}
    f(\phi) &= 0.0167 sen (\phi) - \phi + 1.54927 \\
    f'(\phi) &= 0.0167 cos (\phi) - 1 \\ 
\end{align*}
Con $\phi_1 = 1$
\begin{align*}
    \phi_{1+1} &= 1 - \frac{0.0167 sen (1) - 1 + 1.54927}{0.0167 cos (1) - 1} \\
    \phi_{2} &= 1.568451 \\
\end{align*}
Con $\phi_2 = 1.568451$
\begin{align*}
    \phi_{2+1} &= 1.568451 - \frac{0.0167 sen (1.568451) - 1.568451 + 1.54927}{0.0167 cos (1.568451) - 1} \\
    \phi_{3} &= 1.565969 \\
\end{align*}
Con $\phi_3 = 1.565969$
\begin{align*}
    \phi_{3+1} &= 1.565969 - \frac{0.0167 sen (1.565969) - 1.565969 + 1.54927}{0.0167 cos (1.565969) - 1} \\
    \phi_{4} &= 1.565969 \\
\end{align*}
Esto quiere decir que llegamos a una buena aproximación del valor de $\phi$. \\
\newline
Ahora con la información que tenemos podemos dar el valor de $r$ en $t = 90$ días. \\
Tenemos así:
\begin{align*}
    a &= 150 \times 10^{6} \\
    e &= 0.0167 \\
    \phi &= 1.565969 \\
\end{align*}
Sustituyendo en $r = a(1-ecos \phi)$, tenemos:
\begin{align*}
    r &= 150 \times 10^{6} (1 - 0.0167 cos (1.565969)) \\
    r &= 150 \times 10^{6} (1 - 0.0167 (0.0048273)) \\
    r &= 150 \times 10^{6} (1 - 0.0167 (0.0048273)) \\
    r &= 150 \times 10^{6} (1 - 0.00008061) \\
    r &= 150 \times 10^{6} (0.99991938) \\
    r &= 149987907.6 \\
    r &= 149.9879076 \times 10^{6} \\
\end{align*}
Así sabemos que la distancia de la Tierra al Sol a los 90 días es aproximadamente de $149.9879076 \times 10^{6}$ metros.
