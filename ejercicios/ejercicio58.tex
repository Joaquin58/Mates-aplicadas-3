\chapter*{ANTON-BIVENS-DAVIS 3.6 EJERCICIO 58}

\section*{Problema 58: Verificación de límites que son \\ iguales a $a$}
\begin{enumerate}
    \item[(a)] 
    Resolver el límite:
    \[
    \lim_{x \to 0} \frac{x \ln(a)}{\ln(1 + x)} = a
    \]
    
    \textbf{Solución:}
    
    \begin{itemize}
        \item Primero, reconocemos que al hacer \( x \to 0 \), el límite tiene la forma indeterminada \( \frac{0}{0} \).
        \item Aplicamos la regla de L'Hôpital derivando numerador y denominador:
        \[
        \lim_{x \to 0} \frac{x \ln(a)}{\ln(1 + x)} = \lim_{x \to 0} \frac{\ln(a)}{\frac{1}{1 + x}}.
        \]
        \item Simplificando, obtenemos:
        \[
        \lim_{x \to 0} \ln(a) \cdot (1 + x) = \ln(a) \cdot 1 = \ln(a).
        \]
        \item Finalmente, aplicamos la exponencial para deshacer el logaritmo:
        \[
        e^{\ln(a)} = a.
        \]
    \end{itemize}
    
    Por lo tanto, el resultado es:
    \[
    \lim_{x \to 0} \frac{x \ln(a)}{\ln(1 + x)} = a.
    \]
    
    \item[(b)] 
    Resolver el límite:
    \[
    \lim_{x \to 1} \frac{x \ln(a)}{1 + \ln(x)} = a
    \]
    
    \textbf{Solución:}
    
    \begin{itemize}
        \item Cuando \( x \to 1 \), el numerador se convierte en \( \ln(a) \cdot 1 = \ln(a) \) y el denominador en \( 1 + \ln(1) = 1 \).
        \item Esto nos da:
        \[
        \lim_{x \to 1} \frac{\ln(a)}{1} = \ln(a).
        \]
        \item Aplicando la exponencial para expresar el resultado en términos de \(a\):
        \[
        e^{\ln(a)} = a.
        \]
    \end{itemize}
    
    Por lo tanto, el resultado es:
    \[
    \lim_{x \to 1} \frac{x \ln(a)}{1 + \ln(x)} = a.
    \]
    
    \item[(c)] 
    Resolver el límite:
    \[
    \lim_{x \to 0} \frac{(x + 1)^{\ln(a)}}{x} = a
    \]
    
    \textbf{Solución:}
    
    \begin{itemize}
        \item Tomamos el logaritmo natural de ambos lados para simplificar el exponente. Sea \(L\) el límite:
        \[
        L = \lim_{x \to 0} \frac{(x + 1)^{\ln(a)}}{x}.
        \]
        \item Tomando logaritmo natural en ambos lados:
        \[
        \ln(L) = \lim_{x \to 0} \frac{\ln(a) \cdot \ln(x + 1)}{x}.
        \]
        \item Observamos que este límite es de la forma \( \frac{0}{0} \) cuando \( x \to 0 \), así que aplicamos L'Hôpital:
        \[
        \ln(L) = \lim_{x \to 0} \frac{\ln(a)}{x + 1} = \ln(a).
        \]
        \item Finalmente, aplicamos la exponencial para deshacer el logaritmo:
        \[
        L = e^{\ln(a)} = a.
        \]
    \end{itemize}
    
    Por lo tanto, el resultado es:
    \[
    \lim_{x \to 0} \frac{(x + 1)^{\ln(a)}}{x} = a.
    \]
    
\end{enumerate}