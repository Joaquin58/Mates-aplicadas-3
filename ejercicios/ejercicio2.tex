\chapter*{ANTON-BIVENS-DAVIS 4 EJERCICIO 48}
\textbf{2)} Utilice la diferenciación implícita para demostrar que una función definida implícitamente por $sen x + cos y = 2y$ tiene un punto crítico siempre que $cos x = 0$. Luego utilice la prueba de la primera o segunda derivada para clasificar estos puntos críticos como máximos o mínimos relativos.

\begin{multicols}{2}
	\noindent
	Aplicación de la derivación implícita
	\begin{align*}
		sen (x)+cos(y)=                                                   & 2y                                      \\
		\frac{d}{dx}\left(sen (x)\right)+\frac{d}{dx}\left(cos(y)\right)= & \frac{d}{dx}\left(2y\right)             \\
		cos (x)+\left(-sen(y)\right)\frac{dy}{dx}=                        & 2\frac{dy}{dx}                          \\
		\left(-sen(y)\right)\frac{dy}{dx}-2\frac{dy}{dx}=                 & -cos (x)                                \\
		\frac{dy}{dx}\left(-sen(y)-2\right)=                              & -cos (x)                                \\
		\frac{dy}{dx}=                                                    & \frac{-cos (x)}{\left(-sen(y)-2\right)} \\
		\frac{dy}{dx}=                                                    & \frac{-cos (x)}{-\left(sen(y)+2\right)} \\
		\frac{dy}{dx}=                                                    & \frac{cos (x)}{\left(sen(y)+2\right)}   \\
	\end{align*}
	\columnbreak\\
	Análisis de púntos críticos
	\begin{align*}
		\therefore \frac{dy}{dx}= & 0\iff cos(x)=0                                          \\
		\therefore                & \text{nuestros puntos críticos ocurren en los valores:} \\
		cos(x)=                   & 0\iff x=\frac{\pi}{2}\pm k\pi, k\in\mathbb{Z}           \\
	\end{align*}
\end{multicols}
\vspace{-20px}
\noindent
Clasificación de los puntos críticos. Para clasificar estos puntos críticos, aplicamos la prueba de la segunda derivada.
\begin{align*}
	\frac{d^2y}{dx^2}\left(\frac{cos (x)}{sen(y)+2}\right)= & \frac{(sen(y)+2)\left(-sen(x)\right)-cos(x)\left(cos(y)\frac{dy}{dx}\right)}{\left(sen(y)+2\right)^2}         \\
	=                                                       & \frac{(sen(y)+2)\left(-sen(x)\right)-cos(x)\cdot cos(y)\cdot\frac{cos(x)}{sen(y)+2}}{\left(sen(y)+2\right)^2} \\
	=                                                       & \frac{(sen(y)+2)^2\left(-sen(x)\right)-cos^2(x)\cdot cos(y)}{\left(sen(y)+2\right)^3}
\end{align*}
Dado que la ecuación $coseno$ es periódica, toma el valor de cero en exactamente estos puntos dentro de su periodo.
$cos\left(\frac{\pi}{2}\right)=0$ y $cos\left(\frac{3\pi}{2}\right)=0$

Para ver esto más claramente:

Evaluamos los puntos donde $cos(x)=0$
\begin{align*}
	\frac{d^2y}{dx^2}\bigg|_{cos (x)=0}= & \frac{(sen(y)+2)^2\left(-sen(x)\right)-cos^2(x)\cdot cos(y)}{\left(sen(y)+2\right)^3} \\
	=                                          & \frac{(sen(y)+2)^2\left(-sen(x)\right)-0\cdot cos(y)}{\left(sen(y)+2\right)^3}                                     \\
	=                                          & -\frac{sen(x)}{\left(sen(y)+2\right)}                                          
\end{align*}
\vspace{-20px}
\begin{multicols}{2}
    Sustutuyendo en $\frac{d^2y}{dx^2}$ para $x=\frac{\pi}{2}$:
    \noindent
    \begin{align*}
        \sin\left(\frac{\pi}{2}\right) = 1 \implies \frac{d^2y}{dx^2} = -\frac{1}{\sin y + 2}
    \end{align*}
    \columnbreak\\
    Sustutuyendo en $\frac{d^2y}{dx^2}$ para $x=\frac{3\pi}{2}$:
    \begin{align*}
        \sin\left(\frac{3\pi}{2}\right) = -1 \implies \frac{d^2y}{dx^2} = \frac{1}{\sin y + 2}
    \end{align*}
\end{multicols}
\noindent
Por lo tanto, para $x = \frac{\pi}{2}+k\pi$ es negativo, lo que implica un \textbf{máximo relativo}.\\
Para $x = \frac{3\pi}{2}+k\pi$ es positivo, lo que implica un \textbf{mínimo relativo}.
