
\section*{ANTON-BIVENS-DAVIS 4.7 EJERCICIO 32}

\textbf{Ejercicio 32:} Utilice el método de Newton para aproximar las dimensiones del rectángulo de mayor área que se puede inscribir bajo la curva \( y = \cos(x) \) para \( 0 \leq x \leq \frac{\pi}{2} \).

\section*{Solución}

\textbf{Área del Rectángulo:} Sea \( A(x) \) el área del rectángulo en función de \( x \):
   \begin{equation}
       A(x) = x \cdot \cos(x)
   \end{equation}

\textbf{Derivada de \( A(x) \):} Derivamos \( A(x) \) con respecto a \( x \) para encontrar los puntos críticos.
   \begin{equation}
       A'(x) = \cos(x) - x \sin(x)
   \end{equation}

   Queremos encontrar el valor de \( x \) tal que \( A'(x) = 0 \).

 \textbf{Segunda Derivada de \( A(x) \):} Calculamos \( A''(x) \) para usar el método de Newton.
   \begin{equation}
       A''(x) = -2 \sin(x) - x \cos(x)
   \end{equation}

\textbf{Método de Newton:} La fórmula iterativa del método de Newton es:
   \begin{equation}
       x_n = x_{n-1} - \frac{A'(x_n)}{A''(x_n)}
   \end{equation}

\textbf{Cálculo Numérico:} Empezamos con un valor inicial \( x_0 = 0.5 \) y aplicamos la fórmula iterativa hasta que la diferencia entre \( x_{n+1} \) y \( x_n \).

aproximaciones: 
  \begin{equation}
       x_1 = 0.5 - 1 - \frac{0.9955}{-0.5174} = 1.4240
   \end{equation}
   Iteramos esa fortmula con los valores resultantes anteriores hasta llegar a una amproximación valida.

   \begin{equation}
       x_2 = 1.0765
   \end{equation}

   \begin{equation}
       x_3 = 0.9549
    \end{equation}

    \begin{equation}
       x_3 = 0.9510
    \end{equation}


\textbf{Resultado Final:} El valor de \( x \) que maximiza el área es aproximadamente:
   \begin{equation}
       x_3 = 0.9510
   \end{equation}

\section*{ANTON-BIVENS-DAVIS 4.7 EJERCICIO 33}

\textbf{Ejercicio 33:} Demuestre que en un círculo de radio \( r \), el ángulo central \( \theta \) que
subtiende un arco cuya longitud es \( 1.5 \) veces la longitud \( L \) de
su cuerda satisface la ecuación:
\[
\theta = 3 \sin\left(\frac{\theta}{2}\right)
\]



\section*{Solución}

Sabemos que la longitud del arco \( s \) que subtiende un ángulo central \( \theta \) en un círculo de radio \( r \) está dada por:
   \begin{equation}
       s = r \theta
   \end{equation}

 Longitud de la Cuerda: La longitud \( L \) de la cuerda que subtiende el mismo ángulo \( \theta \) en un círculo de radio \( r \) está dada por:
   \begin{equation}
       L = 2r \sin\left(\frac{\theta}{2}\right)
   \end{equation}

   Esta fórmula proviene de la geometría de un triángulo isósceles formado por los dos radios y la cuerda.

En el problema, la longitud del arco es \( 1.5 \) veces la longitud de la cuerda, es decir:
   \begin{equation}
       s = 1.5 \cdot L
   \end{equation}

Sustituimos los valores de \( s \) y \( L \) en la ecuación anterior:
   \begin{align}
       r \theta &= 1.5 \cdot \left( 2r \sin\left(\frac{\theta}{2}\right) \right) \\
       r \theta &= 3r \sin\left(\frac{\theta}{2}\right)
   \end{align}

 Podemos dividir ambos lados de la ecuación por \( r \), siempre que \( r \neq 0 \):
   \begin{equation}
       \theta = 3 \sin\left(\frac{\theta}{2}\right)
   \end{equation}

 Conclusión: por lo tantp queda demostrado que el ángulo central \( \theta \) satisface la ecuación:
   \[
   \theta = 3 \sin\left(\frac{\theta}{2}\right)
   \]

\end{document}
    
