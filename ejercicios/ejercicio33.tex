\chapter*{ANTON-BIVENS-DAVIS 4.7 EJERCICIO 33}

\textbf{Ejercicio 33:} Demuestre que en un círculo de radio \( r \), el ángulo central \( \theta \) que
subtiende un arco cuya longitud es \( 1.5 \) veces la longitud \( L \) de
su cuerda satisface la ecuación:
\[
\theta = 3 \sin\left(\frac{\theta}{2}\right)
\]



\section*{Solución}

Sabemos que la longitud del arco \( s \) que subtiende un ángulo central \( \theta \) en un círculo de radio \( r \) está dada por:
   \begin{equation}
       s = r \theta
   \end{equation}

 Longitud de la Cuerda: La longitud \( L \) de la cuerda que subtiende el mismo ángulo \( \theta \) en un círculo de radio \( r \) está dada por:
   \begin{equation}
       L = 2r \sin\left(\frac{\theta}{2}\right)
   \end{equation}

   Esta fórmula proviene de la geometría de un triángulo isósceles formado por los dos radios y la cuerda.

En el problema, la longitud del arco es \( 1.5 \) veces la longitud de la cuerda, es decir:
   \begin{equation}
       s = 1.5 \cdot L
   \end{equation}

Sustituimos los valores de \( s \) y \( L \) en la ecuación anterior:
   \begin{align}
       r \theta &= 1.5 \cdot \left( 2r \sin\left(\frac{\theta}{2}\right) \right) \\
       r \theta &= 3r \sin\left(\frac{\theta}{2}\right)
   \end{align}

 Podemos dividir ambos lados de la ecuación por \( r \), siempre que \( r \neq 0 \):
   \begin{equation}
       \theta = 3 \sin\left(\frac{\theta}{2}\right)
   \end{equation}

 Conclusión: por lo tantp queda demostrado que el ángulo central \( \theta \) satisface la ecuación:
   \[
   \theta = 3 \sin\left(\frac{\theta}{2}\right)
   \]
    
