\chapter*{ANTON-BIVENS-DAVIS Cap 4 EJERCICIO 60}
\section*{Problema 60 (Ventana de Iglesia)}

\begin{enumerate}
    \item Sea $r = $ radio del semicírculo\\
    Sea $h = $ altura del rectángulo\\
    Sea $P = $ perímetro total\\
    Base del rectángulo $= 2r$
    
    \item Relaciones de área:
    \begin{itemize}
        \item Área del semicírculo $= \frac{\pi r^2}{2}$ (vidrio azul)
        \item Área del rectángulo $= 2r \times h$ (vidrio transparente)
    \end{itemize}
    
    \item Dado que el vidrio azul es el doble del transparente:
    \[\frac{\pi r^2}{2} = 2(2rh)\]
    \[\frac{\pi r^2}{2} = 4rh\]
    \[h = \frac{\pi r}{8}\]
    
    \item Cálculo del perímetro:
    \begin{align*}
        P &= \text{arco del semicírculo} + \text{base del rectángulo} + \text{lados del rectángulo}\\
        P &= \pi r + 2r + 2h\\
        P &= \pi r + 2r + 2(\frac{\pi r}{8})\\
        P &= \pi r + 2r + \frac{\pi r}{4}\\
        P &= (\frac{5\pi}{4} + 2)r
    \end{align*}
    
    \item Despejamos $r$:
    \[r = \frac{P}{\frac{5\pi}{4} + 2}\]
\end{enumerate}